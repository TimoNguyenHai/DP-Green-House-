\documentclass{article}
\usepackage{array}
\usepackage{textcomp}
\usepackage{hyperref}
\usepackage{color}   %May be necessary if you want to color links
\usepackage{hyperref}
\hypersetup{
    colorlinks=true, %set true if you want colored links
    linktoc=all,     %set to all if you want both sections and subsections linked
    linkcolor=blue,  %choose some color if you want links to stand out
}


\usepackage{siunitx}
\newcolumntype{P}[1]{>{\centering\arraybackslash}p{#1}}
\begin{document}
\tableofcontents
\begin{center}
\section{Introduction}
\LaTeX{} \cite{WinNT} is a set of macros built atop \TeX{} \cite{WinNT}.
\section{Next title}

\begin{tabular}{ | P{0.5cm} | P{2.5cm}| P{1.5cm} | P{1.5cm} | P{1.5cm} | P{1.5cm} |} 
  \hline
  \textnumero & Student initials & $F_1[m^{2}]$ & $F_2[m^{2}]$ & $k_{11}$ [m/s] & $k_{22}$ [m/s]\\ 
  \hline
  1 & RF & 0.1 & 0.8 & 1.0 & 1.3\\ 
  \hline
  2 & HL & 0.1 & 0.8 & 1.3 & 1.0\\
  \hline
  3 & CD & 0.8 & 0.1 & 1.0 & 1.3\\ 
  \hline
  4 & KM & 0.8 & 0.1 & 1.3 & 1.0\\
  \hline
  5 & NHT & 2.0 & 0.8 & 1.0 & 1.3\\ 
  \hline
  6 & MT & 2.0 & 0.8 & 1.3 & 1.0\\
  \hline
  7 & PE & 0.8 & 2.0 & 1.0 & 1.3\\ 
  \hline
  8 & VM & 0.8 & 2.0 & 1.3 & 1.0\\
  \hline
  9 & ZA & 1.2 & 0.1 & 1.0 & 1.3\\ 
  \hline
\end{tabular}
\end{center}
\section{Literature}

\bibliographystyle{plain} % We choose the &quot;plain&quot; reference style
\bibliography{refs} % Entries are in the &quot;refs.bib&quot; file</code></pre>
\end{document}

